\documentclass[ngerman,fontsize=11pt, paper=a4, parskip=half, titlepage=true, toc=bib]{scrartcl}

% DEFINITIONS etc.
% % % % % % % % % % % % % % % % % % % % % %
%
%     AUSARBEITUNG 
%  zum Seminar K-Theory
%       im SS2015
% Universität Regensburg
%  
%     THEMA: Das zahme Symbol und diskrete Bewertungsringe
%            (K2 eines endlichen Körpers veschwindet, diskrete
%             Bewertungsringe und das zahme Symbol [4, Lemma 2.1], die
%             spaltende exakte Milnor-Folge [4, Theorem 2.3].)
%
%
%	Config Datei: Definitionen etc.
%
% % % % % % % % % % % % % %


% ESSENTIAL PACKAGES
% ---------------------------------------------------------------
\usepackage[T1]{fontenc}
\usepackage[utf8]{inputenc}
\usepackage{babel}

\usepackage{lmodern}
\usepackage{microtype}

% EXTRA PACKAGES
% ---------------------------------------------------------------
%% pagestyle
\usepackage{scrlayer-scrpage}
\usepackage{enumerate}
\usepackage{csquotes}

%% maths
\usepackage{amsmath}
\usepackage{mathtools}
\usepackage{amsthm}
\usepackage{amssymb}
\usepackage{dsfont}

%% index/bibliography
\usepackage{makeidx}
\makeindex
\usepackage[date=iso8601,backend=biber,style=alphabetic]{biblatex}
\bibliography{k-theory.bib}

% commutative diagrams
\usepackage{tikz-cd}

%% (nicer)tables
\usepackage{booktabs}
%% hyperlinks
\usepackage{hyperref}


% DEFINITIONS
% ---------------------------------------------------------------
% math (theorems)
\theoremstyle{definition}
\newtheorem{Def}{Definition}[section]
\newtheorem{Bsp}[Def]{Beispiel}
\newtheorem{Not}[Def]{Notation}
% \theoremstyle{remark}
\newtheorem{Bem}[Def]{Bemerkung}
\newtheorem{Wdh}[Def]{Wiederholung}
\theoremstyle{plain}
\newtheorem{Satz}[Def]{Satz}
%\newtheorem*{satz}{Satz}
\newtheorem{Lem}[Def]{Lemma}
\newtheorem{Kor}[Def]{Korollar}
\newtheorem{Fol}[Def]{Folgerung}


% math operators/symbols
\DeclareMathOperator{\Const}{const}
\DeclareMathOperator{\Ker}{ker}
\DeclareMathOperator{\im}{im}
\DeclareMathOperator{\Char}{char}
\DeclareMathOperator{\Ord}{ord}
\DeclareMathOperator{\Quot}{Q}
\DeclareMathOperator{\id}{id}
\DeclareMathOperator{\pr}{pr}
%\DeclareMathOperator{\K}{K}
\newcommand{\K}{K}

\newcommand{\N}{\mathds{N}}
\newcommand{\Z}{\mathds{Z}}
\newcommand{\Q}{\mathds{Q}}
\newcommand{\R}{\mathds{R}}
\newcommand{\C}{\mathds{C}}
\newcommand{\F}{\mathds{F}}
\newcommand{\bF}{\overline{\F}}

\newcommand{\KF}{\K_\ast(\F)}
\newcommand{\KbF}{\K_\ast(\bF)}
\newcommand{\KnF}{\K_n(\F)}
\newcommand{\KnbF}{\K_n(\bF)}
\newcommand{\KoF}{\K_1(\F)}
\newcommand{\Fx}{\F^\times}
\newcommand{\Rx}{R^\times}
\newcommand{\bFx}{\bF^\times}
\newcommand{\Ft}{\F(t)}
\newcommand{\Ftpi}{\F[t]/(\pi)}
\newcommand{\KnFt}{\K_n(\Ft)}
\newcommand{\KnFtpi}{\K_n(\Ftpi)}

\newcommand{\ov}{\overline}

\usepackage{MnSymbol}
\newcommand{\p}{\diamondplus}%{\bigplus}%{\boldsymbol +}}
\newcommand{\m}{\boldsymbol -}

\newcommand{\calA}{\mathcal{A}}
\newcommand{\calK}{\mathcal{K}}
\newcommand{\frakm}{\mathfrak{m}}

\newcommand*{\disoarrow}{\arrow[d,"\rotatebox{90}{$\thicksim$}"]}

\newcommand{\somenote}[1]{\marginpar{#1}}

% STYLE SETTINGS
% ---------------------------------------------------------------
\renewcommand{\theequation}{\thesection.\arabic{equation}}
\ihead{Gesina Schwalbe}
\pagestyle{scrheadings}

\title{Das zahme Symbol und diskrete Bewertungsringe}
\subject{Ausarbeitung}
\subtitle{im Seminar K-Theorie von Prof. Dr. Moritz Kerz}
%\place{Universität Regensburg}
\author{Gesina Schwalbe}
\date{01.07.2015}


%%% Local Variables:
%%% mode: latex
%%% TeX-master: "k-theory_script"
%%% End:



\begin{document}
\maketitle
\tableofcontents

% ------------

\section{Wiederholung Milnor K-Theory}

\begin{Bem}[Gruppentensorprodukt]
  Eine Gruppe $G$ wird mit der Skalarmultiplikation
  \begin{align*}
    \Z \times G &\to G,\\
    (n,g)&\mapsto \underbrace{g+\dotsb+g}_{\scriptsize\text{$n$-mal}}
  \end{align*}
  zu einem $\Z$-Modul.
  Für zwei Gruppen $G$, $H$ ist das 
  Gruppentensorprodukt $G\otimes H$
  definiert als das Tensorprodukt der als $\Z$-Moduln
  aufgefassten Gruppen $G$ und $H$.

  Zu diesem existiert nach Definition eine 
  entsprechende bilineare Abbildung
  \begin{align*}
    \tau\colon G\times H&\rightarrow G\otimes H\\
    \intertext{mit der Eigenschaft}
    \tau(g+g',h) &= \tau(g,h)+\tau(g',h)\\
    \tau(g,h+h') &= \tau(g,h)+\tau(g,h')\\
    \tau(n\cdot g, h) &= n\cdot \tau(g,h)= \tau(g,n\cdot h)\quad
                        \forall n \in\Z
  \end{align*}
  und es erfüllt die universelle Eigenschaft, dass
  \begin{center}
    \begin{tikzcd}
      G\times H 
      \arrow[twoheadrightarrow]{r}{\tau} 
      \arrow[swap]{rd}{\phi}
      & G\otimes H 
      \arrow[dashed]{d}{\exists! \widehat{\phi}}
      \\
      &G'
    \end{tikzcd}
  \end{center}
  für eine beliebige Gruppe $G'$ und einen Gruppenhomomorphismus $\phi$
  kommutiert.

  $G\otimes H$ kann explizit angeben werden als der Quotient
  $(G\times H) / \mathcal{I}$ mit dem Ideal 
  \begin{align*}
    \mathcal{I}\coloneqq \big(
    \big\{&(g+g',h)-(g,h)-(g',h), \\
          &(g,h+h')-(g,h)-(g,h') \colon\\
          &g,g'\in G, h,h'\in H \big\}
            \big)\,.
  \end{align*}
  Im Folgenden wird für eine Gruppe $G$ das $n$-fache Tensorprodukt
  geschrieben als
  \begin{gather*}
    G^{\otimes n}
    \coloneqq \underbrace{G\otimes G\otimes\dotsb\otimes G}_{n\text{-mal}}
  \end{gather*}
\end{Bem}

\subsection{Der Ring $\KF$}
Im Folgenden sei $\F$ ein Körper und $\Fx=\F\setminus \{0\}$ seine
multiplikative Gruppe.

\begin{Def}[$\KF$]\label{defkf}
  Für einen Körper $\F$ definiert man
  \begin{enumerate}[i)]
  \item $\K_0(\F)\coloneqq \Z$
  \item $\K_1(\F)$ ist die multiplikative Gruppe $\Fx$ additiv geschrieben
    vermöge des Gruppenhomomorphismus 
    \begin{align*}
      l\colon \Fx&\to \K_1(\F)\\
      l(ab)&=l(a)\p l(b)
    \end{align*}
  \item $\KnF, n\geq 2,$ ist der Quotient 
    ${(\K_1(\F))}^{\otimes n}/\calA_n$
    des $n$-fachen Gruppentensorprodukts von $\K_1(\F)$ über dem Ideal 
    \begin{gather*}
      \calA_n=\left\{ l(a_1)\otimes\dotsb\otimes l(a_n)
        \in {(\K_1(\F))}^{\otimes n}
        \mid
        \exists 1\leq i < n \colon a_{i+1}=1-a_i, a_i\in\Fx \right\}
    \end{gather*}
  \item $\KF\coloneqq \bigoplus_{n\in\N_0}\KnF$
  \end{enumerate}
\end{Def}


\begin{Bem}\label{identitaetkf}
  Mit der induzierten Addition und dem Tensorprodukt als
  Multiplikation wird die direkte Summe aus $\Z$-Moduln
  \begin{align*}
    \calK &\coloneqq
    \Z\oplus\bigoplus_{n\in\N_0}{\Big(\K_1(\F)\Big)}^{\otimes n}\\
    &\,= \Z\oplus\K_1(\F)
      \oplus\Big(\K_1(\F)\otimes\K_1(\F)\Big)
    \oplus\Big(\K_1(\F)\otimes\K_1\F\otimes\K_1(\F)\Big)
    \oplus\dotsb
  \end{align*}
  zu einer $\Z$-Algebra, 
  ohne Skalarmultiplikation zu einem kommutativen Ring.
  Dieser ist graduiert, d.h. er ist die Summe abelscher Gruppen, 
  der ${\K_1(\F)}^{\otimes n}$, mit einer Multiplikation, hier
  \enquote{$\otimes$}, so dass gilt
  \begin{gather*}
    \forall i,j\in \N\colon 
    \left( {\K_1(\F)}^{\otimes i} \right) \otimes
    \left( {\K_1(\F)}^{\otimes j} \right)
    \subset \left( {\K_1(\F)}^{\otimes ij} \right)
  \end{gather*}
  Die ${\KoF}^{\otimes n}$ können durch Einbettung als additive Untergruppen aufgefasst
  werden, weshalb wir ein Element
  $l(a_1)\otimes\dotsb\otimes l(a_n)\in {\KoF}^{\otimes n}$
  mit seiner Einbettung in $\calK$ 
  $\left(0,0,\dotsc,0,\left(l(a_1)\otimes\dotsb\otimes
      l(a_n)\right),0,\dotsc\right)\in\KF$
  identifizieren werden.
  $\KF$ kann damit als Quotient geschrieben werden:
  \begin{gather*}
    \KF = \calK / \left(l(a)\otimes l(1-a)\mid a\in\Fx\right)
  \end{gather*}
  Analog 
  Diese Identität rührt daher, dass das Herausteilen 
  der $(l(a)\otimes l(1-a))$-Elemente sowohl zuerst in den einzelnen
  Untermoduln stattfinden kann, was die $\KnF$ ergibt, 
  oder nach der Summenbildung.
\end{Bem}

\begin{Bem}
  Die Definitionen in \ref{defkf} können analog auch auf 
  einem kommutativen, unitalen Ring $A$ angewendet werden,
  wobei $\Fx$ der Einheitengruppe entspricht.
\end{Bem}

\begin{Not}
  Der Einfachheit halber wird im Folgenden
  ein Element $a\in\Fx$ mit seinem Bild
  $l(a)\in\K_1(\F)$ identifiziert, wobei wie oben 
  die Addition \enquote{$\p$} in $\KF$
  der Multiplikation \enquote{$\cdot$} in $\F$ entspricht.
  Für Elemente $a_1,a_2,\dotsc,a_n\in\Fx$ wird
  ein Produkt $l(a_1)\otimes\dotsb \otimes l(a_n)\in{\KoF}^{\otimes n}$ 
  mit seiner Äquivalenzklasse in $\KnF$ 
  bzw. dessen Einbettung in $\KF$ identifiziert
  und wir schreiben
  \begin{align*}
    \{ a_1,a_2,\dotsc,a_n\}
    &\coloneqq l(a_1)\otimes l(a_2)\otimes \dotsb \otimes l(a_n) &\in\KF \\
    \intertext{wobei aus den obigen Definitionen folgt}
    \{ a_1 \cdot a_2 \}
    &=  \{a_1\}\p \{a_2\} = l(a_1)\p l(a_2)  &\in\KF \\
    \{ a_1 \} \otimes \{ a_2 \}
    &= \{ a_1,a_2 \} = l(a_1)\otimes l(a_2) &\in\KF \\
    \{ a_1\cdot a_2 \} \otimes \{ a_3 \}
    &= \{ a_1,a_3 \} \p \{a_1, a_2\} 
      =  \big(l(a_1)\p l(a_2)\big) \otimes l(a_3) &\in\KF\\
    \{ a_1, a_2\} 
    &= \m\{a_1^{-1},a_2\} = \m\{a_1,a_2^{-1}\} 
      = \{a_1^{-1},a_2^{-1}\} &\in\KF                                                    
  \end{align*}
\end{Not}

Aus der zusätzlichen Forderung an den Ring $\KF$, 
dass $\{a,1-a\}\equiv 0$ sein soll, ergeben sich neben Nullteilern 
einige weitere interessante Eigenschaften.

\begin{Lem}\label{identitaetminus}
  Für ein $a \in\Fx$ gilt in $\KF$
  \begin{gather*}
    \{a,-a\}=0
  \end{gather*}
  \begin{proof}
    Für den Fall $a=1$ ist $a$ das Nullelement von $\KoF$ und damit ist 
    $\{a,-a\}=l(1)\otimes l(-1) = 0\otimes l(-1) = 0$.
    Falls $a\neq 1$ gilt, kann $-a$ geschrieben werden als
    $-a=\frac{1-a}{1-a^{-1}}$ und es gilt
      \begin{align*}
        \{a,-a\} &= \{a,(1-a)\cdot (1-a^{-1})^{-1}\}\\
                 &= \{a,(1-a)\}\m\{a,(1-a^{-1})\} \\
                 &= \underbrace{\{a,(1-a)\}}_{=0}
                   \p\underbrace{\{a^{-1},(1-a^{-1})\}}_{=0} 
                   =0
      \end{align*}
  \end{proof}
\end{Lem}

\begin{Lem}\label{identitaetmal}
  Für $\chi\in\K_m(\F)$, $\zeta\in\K_n(\F)$ gilt
  \begin{gather*}
    \chi\zeta=(-1)^{mn}\zeta\chi
  \end{gather*}
  Insbesondere ergibt sich für $a,b\in \Fx$
  \begin{gather*}
    \{a,b\}= \m\{b,a\}
  \end{gather*}
  \begin{proof}
    Da alle Elemente in $\KnF$ mit $n>1$ durch Multiplikation aus
    $\KoF$ hervorgehen, genügt es, die Eigenschaft für
    $\chi,\zeta\in\KoF$ zu überprüfen. Dann folgt es wegen
    Assoziativität induktiv für zwei beliebige Elemente aus $\KF$.
    Seien also $\chi=\{a\},\zeta=b\in\Fx$, dann gilt mit \ref{identitaetminus}
    \begin{align*}
      \{a,b\}\p \{b,a\} 
      &= (\overbrace{\{a,-a\}}^{0} \p \{a,b\} )
        \p (\{b,a\} \p \overbrace{\{b,-b\}}^{0} ) \\
      &= \{ a,-ab \} \p \{b, -ab\}\\
      &= \{ ab, -ab\} \overset{\ref{identitaetminus}}{=} 0
    \end{align*}
  \end{proof}
\end{Lem}

\begin{Lem}\label{identitaetquadrat}
  Für jedes $a\in \Fx$ gilt die Identität
  \begin{gather}
    \{a\}^2 = \{a\}\{-1\} = \{-1\}\{a\}
  \end{gather}
  \begin{proof}
    Der Beweis erfolgt wieder durch einfache Rechnung mit
    \ref{identitaetminus}:
    \begin{gather*}
      \{a\}^2 = \{a,(-1)\cdot (-a)\} 
      = \{a,-1\}\p\underbrace{\{a,-a\}}_{0} = \{a,-1\}
    \end{gather*}
    Mit $(-1)=(-1)^{-1}$ und \ref{identitaetmal} folgt die Kommutativität
    \begin{gather*}
      \{a\}^2 = \{a,-1\}= \m\{a,(-1)^{-1}\} = \{(-1)^{-1},a\} = \{-1,a\}
    \end{gather*}
  \end{proof}
\end{Lem}


\begin{Lem}
  Für einen endlichen Körper $\F$ verschwindet $\KnF$ für $n\geq 2$.
  \begin{proof}
    Nachdem alle $\KnF$ für $n>2$ aus $\K_2\F$ durch Multiplikation
    hervorgehen, behandeln wir nur den Fall $n=2$.
    Betrachte nun die Charakteristik $q=\Char(\F)$ von $\F$.

    Ist $q=2$, muss $\F$ wie aus der Algebra bekannt ein
    Zerfällungskörper $k$-ten Grades von $\F_2$ mit entsprechend
    $2^k$ Elementen bzw. $2^k-1$ Elementen in $\Fx$ sein.
    Nach dem kleinen Fermatschen Satz für die multiplikative Gruppe
    $\Fx$ gilt damit $\forall a\in\Fx\colon a^{2^k-1}=1$, 
    d.h. $\forall a\in\Fx\colon a^{2^k} = a$.
    Mit einem Erzeuger $a_0$ von $\Fx$ und zwei Elementen $a=a_0^j,
    b=a_0^i$ ist dann
    \begin{gather*}
      \{a,b\}=\{a_0^i, a_0^j\} = ij\{a_0,a_0\}
      = -ij\{a_0,a_0^{2^k}\} = ij\{a_0, (-a_0)^{2^k}\} =
      ij\{a_0,-a_0\} 
      \overset{\ref{identitaetminus}}{=} 0
    \end{gather*}
  \end{proof}
\end{Lem}


% ------------

\section{Diskrete Bewertungsringe und das zahme Symbol}
\cite[][Lemma 2.1]{milnor}

\subsection{Wiederholung Bewertungen}
\begin{Def}[Bewertungsabbildung]
Eine Bewertung auf einem Körper $\F$ ist eine Abbildung
\begin{gather*}
  v\colon\Fx\to \Gamma
\end{gather*}
von $\Fx$ in eine total geordnete, abelsche Gruppe $(\Gamma, +,\geq)$,
welche für $a,b\in\Fx$ folgende Eigenschaften erfüllt
\begin{enumerate}[i)]
\item $v(ab)=v(a)+v(b)$
\item $v(a+b)\geq \min\{v(a),v(b)\}$
\end{enumerate}
Sie heißt diskret, falls $\Gamma=\Z$.
\end{Def}

\begin{Def}[Diskret bewertete Körper]\end{Def}

\subsection{Das Zahme Symbol in der K-Theorie}
\begin{Satz}[Existenz und Eindeutigkeit eines \enquote{Zahmes
    Symbol}-Homomorphismus]\end{Satz}

% ------------

\section{Die spaltende exakte Milnor-Folge}
\cite[][Theorem 2.3]{milnor}


% ------------------------------------------------------------------

\printindex
% \listoffigures		
% \listoftables

\nocite{*}
\printbibliography


\end{document}

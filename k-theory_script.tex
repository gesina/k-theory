\documentclass[ngerman,fontsize=11pt, paper=a4, parskip=half, titlepage=true, toc=bib]{scrartcl}

% DEFINITIONS etc.
% % % % % % % % % % % % % % % % % % % % % %
%
%     AUSARBEITUNG 
%  zum Seminar K-Theory
%       im SS2015
% Universität Regensburg
%  
%     THEMA: Das zahme Symbol und diskrete Bewertungsringe
%            (K2 eines endlichen Körpers veschwindet, diskrete
%             Bewertungsringe und das zahme Symbol [4, Lemma 2.1], die
%             spaltende exakte Milnor-Folge [4, Theorem 2.3].)
%
%
%	Config Datei: Definitionen etc.
%
% % % % % % % % % % % % % %


% ESSENTIAL PACKAGES
% ---------------------------------------------------------------
\usepackage[T1]{fontenc}
\usepackage[utf8]{inputenc}
\usepackage{babel}

\usepackage{lmodern}
\usepackage{microtype}

% EXTRA PACKAGES
% ---------------------------------------------------------------
%% pagestyle
\usepackage{scrlayer-scrpage}
\usepackage{enumerate}
\usepackage{csquotes}

%% maths
\usepackage{amsmath}
\usepackage{mathtools}
\usepackage{amsthm}
\usepackage{amssymb}
\usepackage{dsfont}

%% index/bibliography
\usepackage{makeidx}
\makeindex
\usepackage[date=iso8601,backend=biber,style=alphabetic]{biblatex}
\bibliography{k-theory.bib}

% commutative diagrams
\usepackage{tikz-cd}

%% (nicer)tables
\usepackage{booktabs}
%% hyperlinks
\usepackage{hyperref}


% DEFINITIONS
% ---------------------------------------------------------------
% math (theorems)
\theoremstyle{definition}
\newtheorem{Def}{Definition}[section]
\newtheorem{Bsp}[Def]{Beispiel}
\newtheorem{Not}[Def]{Notation}
% \theoremstyle{remark}
\newtheorem{Bem}[Def]{Bemerkung}
\newtheorem{Wdh}[Def]{Wiederholung}
\theoremstyle{plain}
\newtheorem{Satz}[Def]{Satz}
%\newtheorem*{satz}{Satz}
\newtheorem{Lem}[Def]{Lemma}
\newtheorem{Kor}[Def]{Korollar}
\newtheorem{Fol}[Def]{Folgerung}


% math operators/symbols
\DeclareMathOperator{\Const}{const}
\DeclareMathOperator{\Ker}{ker}
\DeclareMathOperator{\im}{im}
\DeclareMathOperator{\Char}{char}
\DeclareMathOperator{\Ord}{ord}
\DeclareMathOperator{\Quot}{Q}
\DeclareMathOperator{\id}{id}
\DeclareMathOperator{\pr}{pr}
%\DeclareMathOperator{\K}{K}
\newcommand{\K}{K}

\newcommand{\N}{\mathds{N}}
\newcommand{\Z}{\mathds{Z}}
\newcommand{\Q}{\mathds{Q}}
\newcommand{\R}{\mathds{R}}
\newcommand{\C}{\mathds{C}}
\newcommand{\F}{\mathds{F}}
\newcommand{\bF}{\overline{\F}}

\newcommand{\KF}{\K_\ast(\F)}
\newcommand{\KbF}{\K_\ast(\bF)}
\newcommand{\KnF}{\K_n(\F)}
\newcommand{\KnbF}{\K_n(\bF)}
\newcommand{\KoF}{\K_1(\F)}
\newcommand{\Fx}{\F^\times}
\newcommand{\Rx}{R^\times}
\newcommand{\bFx}{\bF^\times}
\newcommand{\Ft}{\F(t)}
\newcommand{\Ftpi}{\F[t]/(\pi)}
\newcommand{\KnFt}{\K_n(\Ft)}
\newcommand{\KnFtpi}{\K_n(\Ftpi)}

\newcommand{\ov}{\overline}

\usepackage{MnSymbol}
\newcommand{\p}{\diamondplus}%{\bigplus}%{\boldsymbol +}}
\newcommand{\m}{\boldsymbol -}

\newcommand{\calA}{\mathcal{A}}
\newcommand{\calK}{\mathcal{K}}
\newcommand{\frakm}{\mathfrak{m}}

\newcommand*{\disoarrow}{\arrow[d,"\rotatebox{90}{$\thicksim$}"]}

\newcommand{\somenote}[1]{\marginpar{#1}}

% STYLE SETTINGS
% ---------------------------------------------------------------
\renewcommand{\theequation}{\thesection.\arabic{equation}}
\ihead{Gesina Schwalbe}
\pagestyle{scrheadings}

\title{Das zahme Symbol und diskrete Bewertungsringe}
\subject{Ausarbeitung}
\subtitle{im Seminar K-Theorie von Prof. Dr. Moritz Kerz}
%\place{Universität Regensburg}
\author{Gesina Schwalbe}
\date{01.07.2015}


%%% Local Variables:
%%% mode: latex
%%% TeX-master: "k-theory_script"
%%% End:


%\includeonly{chapters/numerik_chap4}


\begin{document}

% FRONTMATTER
% ------------------------------------------------------------------
%\frontmatter
\maketitle
\tableofcontents


% CONTENT
% ------------------------------------------------------------------
%\mainmatter		% part for content

%------------

\section{Wiederholung Milnor K-Theory}

\begin{Bem}[Gruppentensorprodukt]
  Für zwei Gruppen $G$, $H$ ist das 
  \emph{Gruppentensorprodukt} $G\otimes H$ 
  ähnlich dem Tensorprodukt zwischen Moduln definiert
  über den Gruppenhomomorphismus
  \begin{align*}
    \tau\colon G\times H&\rightarrow G\otimes H\\
    \intertext{mit der Eigenschaft}
    \tau(g+g',h) &= \tau(g,h)+\tau(g',h)\\
    \tau(g,h+h') &= \tau(g,h)+\tau(g,h')
  \end{align*}
und die universelle Eigenschaft, dass
\begin{center}
\begin{tikzcd}
  G\times H 
  \arrow[twoheadrightarrow]{r}{\tau} 
  \arrow[swap]{rd}{\phi}
  & G\otimes H 
  \arrow[dashed]{d}{\exists! \widehat{\phi}}
  \\
  &G'
\end{tikzcd}
\end{center}
für eine beliebige Gruppe $G'$ und einen Gruppenhomomorphismus $\phi$
kommutiert.
Sie $G\otimes H$ kann explizit angeben werden als der Quotient
$(G\times H) / \mathcal{I}$ mit dem Ideal 
\begin{align*}
  \mathcal{I}\coloneqq \left(
    \left\{(g+g',h)-(g,h)-(g',h), (g,h+h')-(g,h)-(g,h')
    \mid g,g'\in G, h,h'\in H \right\}
  \right)
\end{align*}
\end{Bem}

\subsection{Der Ring $\KF$}
Im Folgenden sei $\F$ ein Körper und $\Fx=\F\setminus \{0\}$ seine
multiplikative Gruppe.

\begin{Def}[$\KF$]
Für einen Körper $\F$ definiert man
\begin{enumerate}[i)]
\item $\K_0\F\coloneqq \Z$
\item $\K_1\F$ ist die multiplikative Gruppe $\Fx$ additiv geschrieben
  vermöge des Gruppenhomomorphismus 
  \begin{align*}
    l\colon \Fx&\to K_1\F\\
    l(ab)&=l(a)+l(b)
  \end{align*}
\item $\KnF, n\geq 2,$ ist der Quotient 
  \begin{gather*}
    {\K_1\F}^{\otimes n}/\A=\K_1\F\otimes\dotsb\otimes\K_1\F/\A
  \end{gather*}
  mit des $n$-fachen Gruppentensorprodukt von $\K_1\F$ über dem Ideal 
  \begin{gather*}
  \A=\left\{ l(a_1)\otimes\dotsb\otimes l(a_n)\in
    {\K_1\F}^{\otimes n}
    \mid
    \exists 1\leq i\leq n \colon a_i+a_{i+1}=1 \right\}
\end{gather*}
\end{enumerate}
\end{Def}

%------------

\section{Diskrete Bewertungsringe und das zahme Symbol}
\cite[][Lemma 2.1]{milnor}

\subsection{Wiederholung Bewertungen}
\begin{Def}[Bewertungsabbildung]\end{Def}
\begin{Def}[Diskret bewertete Körper]\end{Def}

\subsection{Das klassische Zahme Symbol}
\begin{Def}[klassisches Zahmes Symbol]\end{Def}

\subsection{Das Zahme Symbol in der K-Theorie}
\begin{Satz}[Existenz und Eindeutigkeit eines \enquote{Zahmes
    Symbol}-Homomorphismus]\end{Satz}

%------------

\section{Die spaltende exakte Milnor-Folge}
\cite[][Theorem 2.3]{milnor}

% BACKMATTER
% ------------------------------------------------------------------
%\backmatter
\printindex
% \listoffigures		
% \listoftables

\nocite{*}
\printbibliography


\end{document}

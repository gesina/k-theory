\documentclass[ngerman,fontsize=11pt, paper=a4, parskip=half, titlepage=true, toc=bib]{scrartcl}

% DEFINITIONS etc.
% % % % % % % % % % % % % % % % % % % % % %
%
%     AUSARBEITUNG 
%  zum Seminar K-Theory
%       im SS2015
% Universität Regensburg
%  
%     THEMA: Das zahme Symbol und diskrete Bewertungsringe
%            (K2 eines endlichen Körpers veschwindet, diskrete
%             Bewertungsringe und das zahme Symbol [4, Lemma 2.1], die
%             spaltende exakte Milnor-Folge [4, Theorem 2.3].)
%
%
%	Config Datei: Definitionen etc.
%
% % % % % % % % % % % % % %


% ESSENTIAL PACKAGES
% ---------------------------------------------------------------
\usepackage[T1]{fontenc}
\usepackage[utf8]{inputenc}
\usepackage{babel}

\usepackage{lmodern}
\usepackage{microtype}

% EXTRA PACKAGES
% ---------------------------------------------------------------
%% pagestyle
\usepackage{scrlayer-scrpage}
\usepackage{enumerate}
\usepackage{csquotes}

%% maths
\usepackage{amsmath}
\usepackage{mathtools}
\usepackage{amsthm}
\usepackage{amssymb}
\usepackage{dsfont}

%% index/bibliography
\usepackage{makeidx}
\makeindex
\usepackage[date=iso8601,backend=biber,style=alphabetic]{biblatex}
\bibliography{k-theory.bib}

% commutative diagrams
\usepackage{tikz-cd}

%% (nicer)tables
\usepackage{booktabs}
%% hyperlinks
\usepackage{hyperref}


% DEFINITIONS
% ---------------------------------------------------------------
% math (theorems)
\theoremstyle{definition}
\newtheorem{Def}{Definition}[section]
\newtheorem{Bsp}[Def]{Beispiel}
\newtheorem{Not}[Def]{Notation}
% \theoremstyle{remark}
\newtheorem{Bem}[Def]{Bemerkung}
\newtheorem{Wdh}[Def]{Wiederholung}
\theoremstyle{plain}
\newtheorem{Satz}[Def]{Satz}
%\newtheorem*{satz}{Satz}
\newtheorem{Lem}[Def]{Lemma}
\newtheorem{Kor}[Def]{Korollar}
\newtheorem{Fol}[Def]{Folgerung}


% math operators/symbols
\DeclareMathOperator{\Const}{const}
\DeclareMathOperator{\Ker}{ker}
\DeclareMathOperator{\im}{im}
\DeclareMathOperator{\Char}{char}
\DeclareMathOperator{\Ord}{ord}
\DeclareMathOperator{\Quot}{Q}
\DeclareMathOperator{\id}{id}
\DeclareMathOperator{\pr}{pr}
%\DeclareMathOperator{\K}{K}
\newcommand{\K}{K}

\newcommand{\N}{\mathds{N}}
\newcommand{\Z}{\mathds{Z}}
\newcommand{\Q}{\mathds{Q}}
\newcommand{\R}{\mathds{R}}
\newcommand{\C}{\mathds{C}}
\newcommand{\F}{\mathds{F}}
\newcommand{\bF}{\overline{\F}}

\newcommand{\KF}{\K_\ast(\F)}
\newcommand{\KbF}{\K_\ast(\bF)}
\newcommand{\KnF}{\K_n(\F)}
\newcommand{\KnbF}{\K_n(\bF)}
\newcommand{\KoF}{\K_1(\F)}
\newcommand{\Fx}{\F^\times}
\newcommand{\Rx}{R^\times}
\newcommand{\bFx}{\bF^\times}
\newcommand{\Ft}{\F(t)}
\newcommand{\Ftpi}{\F[t]/(\pi)}
\newcommand{\KnFt}{\K_n(\Ft)}
\newcommand{\KnFtpi}{\K_n(\Ftpi)}

\newcommand{\ov}{\overline}

\usepackage{MnSymbol}
\newcommand{\p}{\diamondplus}%{\bigplus}%{\boldsymbol +}}
\newcommand{\m}{\boldsymbol -}

\newcommand{\calA}{\mathcal{A}}
\newcommand{\calK}{\mathcal{K}}
\newcommand{\frakm}{\mathfrak{m}}

\newcommand*{\disoarrow}{\arrow[d,"\rotatebox{90}{$\thicksim$}"]}

\newcommand{\somenote}[1]{\marginpar{#1}}

% STYLE SETTINGS
% ---------------------------------------------------------------
\renewcommand{\theequation}{\thesection.\arabic{equation}}
\ihead{Gesina Schwalbe}
\pagestyle{scrheadings}

\title{Das zahme Symbol und diskrete Bewertungsringe}
\subject{Ausarbeitung}
\subtitle{im Seminar K-Theorie von Prof. Dr. Moritz Kerz}
%\place{Universität Regensburg}
\author{Gesina Schwalbe}
\date{01.07.2015}


%%% Local Variables:
%%% mode: latex
%%% TeX-master: "k-theory_script"
%%% End:



\begin{document}
\maketitle
\tableofcontents

% ------------

\section{Milnor K-Theorie}

\begin{Bem}[Gruppentensorprodukt]
  Eine abelsche Gruppe $G$ wird mit der Skalarmultiplikation
  \begin{align*}
    \Z \times G &\to G,\\
    (n,g)&\mapsto \underbrace{g+\dotsb+g}_{\scriptsize\text{$n$-mal}}
  \end{align*}
  zu einem $\Z$-Modul.
  Für zwei Gruppen $G$, $H$ ist das 
  Gruppentensorprodukt $G\otimes H$
  definiert als das Tensorprodukt der als $\Z$-Moduln
  aufgefassten Gruppen $G$ und $H$.

  Zu diesem existiert nach Definition eine 
  entsprechende bilineare Abbildung
  \begin{align*}
    \tau\colon G\times H&\rightarrow G\otimes H\\
    \intertext{mit der Eigenschaft (für $g,g'\in G$, $h,h'\in H$)}
    \tau(g+g',h) &= \tau(g,h)+\tau(g',h)\\
    \tau(g,h+h') &= \tau(g,h)+\tau(g,h')\\
    \tau(n\cdot g, h) &= n\cdot \tau(g,h)= \tau(g,n\cdot h)\quad
                        \forall n \in\Z
  \end{align*}
  und es erfüllt die universelle Eigenschaft, dass
  \begin{center}
    \begin{tikzcd}
      G\times H 
      \arrow[twoheadrightarrow]{r}{\tau} 
      \arrow[swap]{rd}{\phi}
      & G\otimes H 
      \arrow[dashed]{d}{\exists! \widehat{\phi}}
      \\
      &G'
    \end{tikzcd}
  \end{center}
  für eine beliebige Gruppe $G'$ und einen eindeutigen Gruppenhomomorphismus $\phi$
  kommutiert.

  $G\otimes H$ kann explizit angeben werden als der Quotient
  $(G\times H) / \mathcal{I}$ mit dem Ideal 
  \begin{align*}
    \mathcal{I}\coloneqq \big(
    \big\{&(g+g',h)-(g,h)-(g',h), \\
          &(g,h+h')-(g,h)-(g,h') \colon\\
          &g,g'\in G, h,h'\in H \big\}
            \big)\,.
  \end{align*}
  Im Folgenden wird für eine Gruppe $G$ das $n$-fache Tensorprodukt
  geschrieben als
  \begin{gather*}
    G^{\otimes n}
    \coloneqq \underbrace{G\otimes G\otimes\dotsb\otimes G}_{n\text{-mal}}
  \end{gather*}
\end{Bem}

\subsection{Der Ring $\KF$}
Im Folgenden sei $\F$ ein Körper und $\Fx=\F\setminus \{0\}$ seine
multiplikative Gruppe.

\begin{Def}[$\KF$]\label{defkf}
  Für einen Körper $\F$ definiert man
  \begin{enumerate}[a)]
  \item $\K_0(\F)\coloneqq \Z$
  \item $\K_1(\F)$ ist die multiplikative Gruppe $\Fx$ 
    additiv geschrieben mit Addition $\p$
    vermöge des Gruppenhomomorphismus 
    \begin{align*}
      l\colon \Fx&\to \K_1(\F)\\
      l(ab)&=l(a)\p l(b)
    \end{align*}
  \item $\KnF, n\geq 2,$ ist der Quotient 
    ${(\K_1(\F))}^{\otimes n}/\calA_n$
    des $n$-fachen Gruppentensorprodukts von $\K_1(\F)$ über dem Ideal 
    \begin{gather*}
      \calA_n=\left\{ l(a_1)\otimes\dotsb\otimes l(a_n)
        \in {(\K_1(\F))}^{\otimes n}
        \mid  a_i\in\Fx,
        \exists 1\leq i < n \colon a_{i+1}=1-a_i \right\}
    \end{gather*}
  \item $\KF\coloneqq \bigoplus_{n\in\N_0}\KnF$
  \end{enumerate}
\end{Def}


\begin{Bem}\label{identitaetkf}
  Mit der induzierten Addition und dem Tensorprodukt als
  Multiplikation wird die direkte Summe aus $\Z$-Moduln
  \begin{align*}
    \calK &\coloneqq
            \Z\oplus\bigoplus_{n\in\N_0}{\Big(\K_1(\F)\Big)}^{\otimes n}\\
          &\,= \Z\oplus\K_1(\F)
            \oplus\Big(\K_1(\F)\otimes\K_1(\F)\Big)
            \oplus\Big(\K_1(\F)\otimes\K_1\F\otimes\K_1(\F)\Big)
            \oplus\dotsb
  \end{align*}
  zu einer $\Z$-Algebra, 
  ohne Skalarmultiplikation zu einem kommutativen Ring mit Einselement
  $1\in\Z$.
  Dieser ist graduiert, d.h. er ist die Summe abelscher Gruppen, 
  der ${\K_1(\F)}^{\otimes n}$, mit einer Multiplikation, hier
  \enquote{$\otimes$}, so dass gilt
  \begin{gather*}
    \forall i,j\in \N\colon 
    \left( {\K_1(\F)}^{\otimes i} \right) \otimes
    \left( {\K_1(\F)}^{\otimes j} \right)
    \subset \left( {\K_1(\F)}^{\otimes ij} \right)
  \end{gather*}
  Die ${\KoF}^{\otimes n}$ können durch Einbettung als additive 
  Untergruppen von $\calK$ aufgefasst
  werden, weshalb wir ein Element
  $l(a_1)\otimes\dotsb\otimes l(a_n)\in {\KoF}^{\otimes n}$
  vorerst mit seiner Einbettung
  $\left(0,0,\dotsc,0,\left(l(a_1)\otimes\dotsb\otimes
      l(a_n)\right),0,\dotsc\right)\in\calK$
  identifizieren.
  $\KF$ kann damit als Quotient geschrieben werden:
  \begin{gather*}
    \KF = \calK / \left(l(a)\otimes l(1-a)\mid a\in\Fx\right)
  \end{gather*}
  Diese Identität rührt daher, dass das Herausteilen 
  der $(l(a)\otimes l(1-a))$-Elemente sowohl zuerst in den einzelnen
  Untermoduln stattfinden kann, was die $\KnF$ als Quotienten der
  $\KoF^{\otimes n}$ ergibt, 
  oder nach der Summenbildung, wie hier.
\end{Bem}

\begin{Bem}
  Die Definitionen in \ref{defkf} und die Äquivalenz in 
  \ref{identitaetkf} können analog auch auf 
  einem kommutativen, unitalen Ring $A$ angewendet werden,
  wobei $\Fx$ der Einheitengruppe entspricht. Siehe \cite{kerzdipl,kerz}.
\end{Bem}

\begin{Not}
  Der Einfachheit halber wird im Folgenden
  ein Element $a\in\Fx$ mit seinem Bild
  $l(a)\in\K_1(\F)$ identifiziert, wobei wie oben 
  die Addition \enquote{$\p$} in $\KF$
  der Multiplikation \enquote{$\cdot$} in $\F$ entspricht.
  Für Elemente $a_1,a_2,\dotsc,a_n\in\Fx$ wird
  ein Produkt $l(a_1)\otimes\dotsb \otimes l(a_n)\in{\KoF}^{\otimes n}$ 
  mit seiner Äquivalenzklasse in $\KnF$ 
  bzw. dessen Einbettung in $\KF$ identifiziert
  und wir schreiben
  \begin{align*}
    \{ a_1,a_2,\dotsc,a_n\}
    &\coloneqq \overline{l(a_1)\otimes l(a_2)\otimes \dotsb \otimes l(a_n)} &\in\KF 
                                                                              \intertext{wobei aus den obigen Definitionen folgt}
                                                                              \{1\}&=0 &\in\KF\\
    \{ a_1 \cdot a_2 \}
    &=  \{a_1\}\p \{a_2\} = \overline{l(a_1)} \p \overline{l(a_2)}  &\in\KF \\
    \{ a_1 \} \otimes \{ a_2 \}
    &= \{ a_1,a_2 \} = \overline{l(a_1)\otimes l(a_2)} &\in\KF \\
    \{ a_1\cdot a_2 \} \otimes \{ a_3 \}
    &= \{ a_1,a_3 \} \p \{a_1, a_2\} 
      = \overline{\big(l(a_1)\p l(a_2)\big) \otimes l(a_3)} &\in\KF\\
    \{ a_1, a_2\} 
    &= \m\{a_1^{-1},a_2\} = \m\{a_1,a_2^{-1}\} 
      = \{a_1^{-1},a_2^{-1}\} &\in\KF                               
  \end{align*}
\end{Not}

\subsection{Identitäten auf $\KF$}
Aus der zusätzlichen Forderung an den Ring $\KF$, 
dass $\{a,1-a\}\equiv 0$ sein soll, ergeben sich neben Nullteilern 
einige weitere interessante Eigenschaften der Elemente.

\begin{Lem}\label{identitaetminus}
  Für ein $a \in\Fx$ gilt in $\KF$
  \begin{gather*}
    \{a,-a\}=0
  \end{gather*}
  \begin{proof}
    Für den Fall $a=1$ ist $a$ das Nullelement von $\KoF$ und damit ist 
    $\{a,-a\}=l(1)\otimes l(-1) = 0\otimes l(-1) = 0$.
    Falls $a\neq 1$ gilt, kann $-a$ geschrieben werden als
    $-a=\frac{1-a}{1-a^{-1}}$ und es gilt
    \begin{align*}
      \{a,-a\} &= \{a,(1-a)\cdot (1-a^{-1})^{-1}\}\\
               &= \{a,(1-a)\}\p\{a,(1-a^{-1})^{-1}\} \\
               &= \{a,(1-a)\}\m\{a,(1-a^{-1})\} \\
               &= \underbrace{\{a,(1-a)\}}_{=0}
                 \p\underbrace{\{a^{-1},(1-a^{-1})\}}_{=0} 
                 =0
    \end{align*}
  \end{proof}
\end{Lem}

\begin{Lem}\label{identitaetmal}
  Für $\chi\in\K_m(\F)$, $\zeta\in\K_n(\F)$ gilt
  \begin{gather*}
    \chi\zeta=(-1)^{mn}\zeta\chi
  \end{gather*}
  Insbesondere ergibt sich für $a,b\in \Fx$
  \begin{gather*}
    \{a,b\}= \m\{b,a\}
  \end{gather*}
  \begin{proof}
    Da alle Elemente in $\KnF$ mit $n>1$ durch Multiplikation aus
    $\KoF$ hervorgehen, genügt es, die Eigenschaft für
    $\chi,\zeta\in\KoF$ zu überprüfen. Dann folgt es wegen
    Assoziativität induktiv für zwei beliebige Elemente aus $\KF$.
    Seien also $\chi=\{a\},\zeta=b\in\Fx$, dann gilt mit \ref{identitaetminus}
    \begin{align*}
      \{a,b\}\p \{b,a\} 
      &= (\overbrace{\{a,-a\}}^{0} \p \{a,b\} )
        \p (\{b,a\} \p \overbrace{\{b,-b\}}^{0} ) \\
      &= \{ a,-ab \} \p \{b, -ab\}\\
      &= \{ ab, -ab\} \overset{\ref{identitaetminus}}{=} 0
    \end{align*}
  \end{proof}
\end{Lem}

\begin{Lem}\label{identitaetquadrat}
  Für jedes $a\in \Fx$ gilt die Identität
  \begin{gather}
    \{a\}^2 = \{a\}\{-1\} = \{-1\}\{a\}
  \end{gather}
  \begin{proof}
    Der Beweis erfolgt wieder durch einfache Rechnung mit
    \ref{identitaetminus}:
    \begin{gather*}
      \{a\}^2 = \{a,(-1)\cdot (-a)\} 
      = \{a,-1\}\p\underbrace{\{a,-a\}}_{0} = \{a,-1\}
    \end{gather*}
    Mit $(-1)=(-1)^{-1}$ und \ref{identitaetmal} folgt die Kommutativität
    \begin{gather*}
      \{a\}^2 = \{a,-1\}= \m\{a,(-1)^{-1}\} = \{(-1)^{-1},a\} = \{-1,a\}
    \end{gather*}
  \end{proof}
\end{Lem}


\subsection{$\K_2$ eines endlichen Körpers verschwindet}

\begin{Lem}
  Für einen endlichen Körper $\F$ verschwindet $\KnF$ für $n\geq 2$.
  \begin{proof}
    Nachdem alle $\KnF$ für $n>2$ aus $\K_2\F$ durch Multiplikation
    hervorgehen, behandeln wir nur den Fall $n=2$.
    Betrachte nun die Charakteristik $q=\Char(\F)$ von $\F$.

    Ist $q=2$, muss $\F$ wie aus der Algebra bekannt ein
    Zerfällungskörper $k$-ten Grades von $\F_2$ mit entsprechend
    $q^k=2^k$ Elementen bzw. $2^k-1$ Elementen in $\Fx$ sein und $\Fx$
    ist zyklische Gruppe.
    Nach dem kleinen Fermatschen Satz für die multiplikative Gruppe
    $\Fx$ mit $\Ord(\Fx)=2^k-1$ gilt damit $\forall a\in\Fx\colon a^{2^k-1}=1$, 
    d.h. $\forall a\in\Fx\colon a^{2^k} = a$.
    Mit einem Erzeuger $a_0$ von $\Fx$ und zwei beliebigen Elementen $a=a_0^j,
    b=a_0^i$ ist dann
    \begin{gather*}
      \{a,b\}=\{a_0^i, a_0^j\} = ij\{a_0,a_0\}
      = -ij\{a_0,a_0^{2^k}\} = ij\{a_0, (-a_0)^{2^k}\} =
      ij\{a_0,-a_0\} 
      \overset{\ref{identitaetminus}}{=} 0
    \end{gather*}

    Falls $q>2$ kann genutzt werden, dass $\Char(\K_2\F)\leq 2$, 
    d.h. $2\{a,b\}=0\;\forall a,b\in\Fx$,
    denn für einen Erzeuger $a_0\in\Fx$ und
    Elemente $a=a_0^i,b=a_0^j\in \Fx$ ist
    \begin{gather*}
      \{a,b\} = \{a_0^i,a_0^j\} = ij\{a_0,a_0\} 
      \overset{\ref{identitaetmal}}{=} \m ij\{a_0,a_0\}
      = \m \{a_0^i,a_0^j\} = \m\{a,b\} 
    \end{gather*}
    Damit verschwinden bereits alle Elemente der Form 
    $\{a^2,b\}=2\{a,b\}=\{a,b^2\}$. Es bleibt demnach zu zeigen, dass
    auch $\{a,b\}$ mit Nichtquadraten $a,b\in\Fx$
    (d.h. $\not\exists a',b'\in\Fx\colon (a')^2=a,(b')^2=b$)
    verschwinden.
    Hierfür reicht es – wie wir gleich sehen werden – aus,
    $\{a,b\}$ mit fixen Nichtquadraten
    $a,b\in\Fx$ zu betrachten, die wir mit $b=1-a$ wählen.
    
    Betrachte vorerst den Gruppenhomomorphismus
    \begin{gather*}
      \phi\colon\Fx \to \Fx
      \qquad x\mapsto x^2
    \end{gather*}
    Es ist offensichtlich $\Ker(\phi)=\{1,-1\}$ (einzige Lösungen von
    $x^2=1$), also ist nach dem Homomorphiesatz wegen $\phi$ surjektiv
    $\im(\phi)\cong \Fx/\{1,-1\}$. Damit ist die Anzahl der Quadrate
    in $\Fx$ gleich der Ordnung von $\Fx/\{1,-1\}$, die nach dem Satz
    von Lagrange
    $\frac{\Ord(\Fx)}{\Ord(\{1,-1\})}=\frac{1}{2}\Ord(\Fx)$
    beträgt, d.h. die Hälfte aller Elemente sind Nichtquadrate
    bzw. Quadrate. Beachte hierbei, dass $-1\neq 1$ wegen $\Char(\F)>2$.

    Jetzt brauchen wir noch, dass das Produkt zweier Nichtquadrate ein
    Quadrat ist. Das folgt daraus, dass  die Menge 
    $\{x^2\mid x\in\Fx\}$ der Quadrate 
    als Bild eines surjektiven Homomorphismus eine Untergruppe ist und 
    \begin{gather*}
      \Ord(\Fx/\{x^2\mid x\in\Fx\})
      =\frac{\Ord(\Fx)}{\Ord(\{x^2\mid x\in\Fx\})}
      =\frac{\Ord(\Fx)}{\frac{1}{2}\Ord(\Fx)}=2
      \quad\text{(wieder mit Lagrange).}
    \end{gather*} 
    Also ist $\Fx/\{x^2\mid x\in\Fx\}\cong \Z/2\Z$ und das Produkt 
    zweier Nichtquadrate wird im Quotienten
    $\Fx/\{x^2\mid x\in\Fx\}$ Null, weshalb es ein Quadrat in $\Fx$
    sein muss.

    Damit können wir nun zeigen, dass fixe Nichtquadrate $a,b\in\Fx$
    ausreichen, denn es gilt für weitere Nichtquadrate $c,d\in\Fx$
    \begin{gather*}
      \{c,d\}=\{(ca^{-1})a,(db^{-1})b\}=\{g^2a,h^2b\}
      =\{a,b\}\p\{g^2,h^2\}\p\{g^2,b\}\p\{a,h^2\}
      =\{a,b\}
    \end{gather*}
    wobei $ca^{-1}=g^2,db^{-1}=h^2$ für bestimmte $g,h\in\Fx$
    wegen $a^{-1},b^{-1}$ Nichtquadrate.

    Um zu zeigen, dass Nichtquadrate $a,b$ mit der gewünschten
    Eigenschaft $b=a-1$ existieren, wende auf $\Fx\setminus\{1\}$ 
    folgende Bijektion an
    \begin{align*}
      \Fx\setminus\{1\}&\to \Fx\setminus\{1\}
      &x\mapsto 1-x
    \end{align*}
    Wegen der Überzahl der Nichtquadrate um 1 muss gemäß simpler
    Kombinatorik wenigstens ein Nichtquadrat $a$  auf ein Nichtquadrat
    $1-a=b$ abgebildet werden, was die Existenz
    von $a,b\in\Fx$ wie oben zeigt.

    Für beliebige Nichtquadrate $c,d\in\Fx$ gilt nun:
    \begin{gather*}
      \{c,d\}=\{a,b\}=\{a,1-a\} \overset{\text{Def}}{=} 0
    \end{gather*}
    Damit sind alle Elemente $\{a,b\}\in\Fx$, sowohl mit $a,b$
    Nichtquadrate als auch mit einem oder beiden ein Quadrat, 
    in $\K_2\F$ bereits Null und $\K_2\F$
    bzw. damit alle $\K_n\F,n\geq 2,$ verschwinden für einen endlichen
    Körper $\F$.
  \end{proof}
\end{Lem}



% ------------

\section{Grundlagen aus der Bewertungstheorie}

\subsection{Bewertungsringe und Bewertungen}
% \begin{Def}[Bewertungsabbildung]
%   Eine Bewertung auf einem Körper $\F$ ist eine Abbildung
%   \begin{gather*}
%     v\colon\F\to \Gamma\cup \{\infty\}
%   \end{gather*}
%   wobei $(\Gamma, +,\geq)$ eine total geordnete, abelsche Gruppe ist
%   und $v$ für $a,b\in\F$ folgende Eigenschaften erfüllt
%   \begin{enumerate}[(1)]
%   \item $v(a)=\infty \Longleftrightarrow a=0$
%   \item $v(ab)=v(a)+v(b)$~~(d.h. $v|_{\Fx}$ ist Gruppenhomomorphismus)
%   \item $v(a+b)\geq \min\{v(a),v(b)\}$
%   \end{enumerate}
%   Sie heißt diskret, falls $\Gamma=\Z$ und $v$ surjektiv
%   (d.h. insbesondere $v^{-1}(1)\neq \varnothing$).
% \end{Def}

% \begin{Bsp}
%   Ein bekanntes Beispiel, das auch bereits in der Algebra eingeführt
%   wurde, ist die sog. $p$-adische Bewertung $v_p$ auf $\Q$ zu einem
%   Primelement $p\in\Z$. Sie ist für $\frac{a}{b}\in\Q, a,b\in\Z$
%   gegeben als die Differenz der Primpotenzen von $p$ in den
%   Primfaktorzerlegungen von $a$ und $b$.
%   Also wäre z.B. $v_3(\frac{15}{9}=v_3(\frac{3^1\cdot 5^1}{3^2})=1-2=-1$.
% \end{Bsp}

\begin{Def}[Bewertungsring]\label{defbewertungsring}
  Ein Integritätsring $R$ mit Quotientenkörper $\Quot(R)=\F$ heißt
  Bewertungsring, falls $\forall x\in\F\colon x\in\R$ oder
  $x^{-1}\in\R$.
  Wir sagen auch $R$ ist Bewertungsring zum Körper $\F$.
\end{Def}

\begin{Lem}[Eigenschaften Bewertungsringe]\label{eigbewertungsring}
  Sei $R$ Bewertungsring zum Körper $\F$, dann gilt
  \begin{enumerate}[i)]
  \item Die Ideale von $R$ sind total geordnet bzgl. 
    \enquote{$\subset$} und $R$ ist lokal.
  \item Es gibt eine zu $R$ bzw. $\F$ zugehörige Abbildung
    $v\colon\F\to\Gamma\cup\{\infty\}$ in die total geordnete, abelsche Gruppe
    $\Gamma\coloneqq \Fx/R^\times$ vereinigt mit ihrem Supremum $\infty$, wobei $v$ erfüllt
    \begin{enumerate}[(1)]
    \item $v(a)=\infty \Longleftrightarrow a=0$
    \item $v(ab)=v(a)+v(b)$~~(d.h. $v|_{\Fx}$ ist Gruppenhomomorphismus)
    \item $v(a+b)\geq \min\{v(a),v(b)\}$
    \end{enumerate}
    Eine solche Abbildung, die i)-iii) erfüllt,
    wird Bewertungsabbildung genannt.
    $R$ heißt auch Bewertungsring zur Bewertung $v$.
  \item $v$ ist bis auf Verknüpfung mit ordnungserhaltenden 
    Gruppenisomorphismen auf $\Gamma$ eindeutig.
    Ein entsprechendes $v$ ist über $v|_{R}$ bereits eindeutig 
    bestimmt.
  \end{enumerate}

  \begin{proof}
    \begin{enumerate}[i)]
    \item Wir zeigen, dass für zwei Ideale $I\neq J\subset R$ bereits
      eines das andere enthalten muss. Es gebe also OBdA ein $x\in I$
      mit $x\not\in J$. Für ein $y\in J$ ist $\frac{x}{y}\not\in\R$,
      da sonst $y\cdot \frac{x}{y}=x\in J$ gelten würde.
      Weil $R$ Bewertungsring ist, muss aber
      $(\frac{x}{y})^{-1}=\frac{y}{x}\in R$ gelten und $y=x\cdot
      \frac{y}{x}\in I$. Damit muss $J\subsetneq I$ sein und die
      Ideale total geordnet bzgl. \enquote{$\subset$}.
      Weil es nur ein maximales Element bzgl. einer totalen Ordnung
      gibt, kann $R$ nur ein maximales Ideal haben und ist somit
      lokal.
    \item Für die Konstruktion einer solchen Bewertungsabbildung
      benötigen wir die totale Ordnung auf der entsprechenden Gruppe $\Gamma$.
      Es ist
      \begin{gather*}
        \Gamma \coloneqq \left\{ xR \mid x\in \Fx \right\}\cong\Fx/R^{\times}
      \end{gather*}
      mit der Verknüpfung $xR\cdot yR=(xy)R$ (später additiv
      geschrieben) eine Gruppe. Die umgekehrte Ordnungsrelation
      auf $\Gamma$
      \begin{gather*}
        xR\leq yR \Longleftrightarrow xR \supset yR
      \end{gather*}
      ist wegen i) total, denn die Ordnung der Ideale ist bereits total.
      Definieren wir nun $v$ über
      \begin{align*}
        \forall x\in \Fx\colon\;v(x)&\coloneqq xR\in\Gamma
        &v(0)&\coloneqq \infty
      \end{align*}
      ist $v$ eine Bewertungsabbildung. 
    \item Dass ein entsprechendes $v$ aus ii) bereits eindeutig
      durch $v|_{R}$ definiert ist, folgt leicht aus
      \begin{gather*}
        v(x)+v(x^{-1}) = v(x\cdot x^{-1}) = v(1) = R  =
        \text{Nullelement in }\Gamma.
        \Longrightarrow v(x^{-1})=-v(x)
      \end{gather*}
      und $x\in R$ oder $x^{-1}\in R$ für alle $x\in K$ wegen $R$
      Bewertungsring.
      Die Eindeutigkeit von $v$ bis auf Isomorphie folgt aus der
      Konstruktion.
    \end{enumerate}
  \end{proof}
\end{Lem}


Tatsächlich ist ein Integritätsring genau dann
ein Bewertungsring nach Definition \ref{defbewertungsring},
wenn man eine Bewertungsabbildung $v$ zu $R$ findet,
wie wir jetzt sehen werden.

\begin{Lem}\label{bewertungsringausbewertung}
  Sei $\Fx$ ein Körper mit einer Bewertungsabbildung
  $v\colon\F\to\Gamma\cup\{\infty\}$ in eine total geordnete, abelsche Gruppe
  $\Gamma$. Dann gilt:
  \begin{enumerate}[i)]
  \item   Die Menge $R\coloneqq\left\{x\in \F\mid  v(x)\geq 0\right\}$ bildet
    einen Bewertungsring zu $\F$ (d.h. $\Quot(R)=\F$).
  \item Die Einheitengruppe von $R$ ist 
    $R^\times\coloneqq\left\{x\in\F\mid v(x)=0\right\}$
  \item Das maximale Ideal von $R$ ist
    $\frakm_v\coloneqq\left\{x\in\F\mid v(x)>1\right\}$
  \end{enumerate}

  \begin{proof}
    \begin{enumerate}[i)]
    \item $R$ ist mit unter Multiplikation und Addition abgeschlossen, da für
      $x,y\in R$, d.h. $v(x),v(y)\geq 0$, 
      gilt $v(xy)=v(x)+v(y)\geq 0$ und 
      $v(x+y)\geq \min\{v(x),v(y)\}\geq 0$, also $xy, x+y\in R$.
      Es ist auch $1\in R$ wegen $v(1)=0\geq 0$.
      $R$ wird somit  zum Unterring von $\F$ und ist als solcher integer.
      Außerdem ist für ein $x\in \F$ entweder $v(x)<0$
      und damit $v(x^{-1})=-v(x)>0$ also $x^{-1}\in R$ oder es ist
      bereits $v(x)\geq 0$, d.h. $x\in R$.
      Wir erhalten, dass $R$ nach Definition \ref{defbewertungsring}
      ein Bewertungsring ist.
    \item Für eine Einheit $x\in R$ muss gelten, dass $x\in R$,
      d.h. $v(x)\geq 0$, und $x^{-1}\in R$, d.h. $v(x^{-1})=-v(x)\geq 0$. 
      $x$ erfüllt also $v(x)\leq 0 \geq v(x)$, woraus bereits $v(x)=0$
      folgt.
      Umgekehrt genauso, denn für $x\in R$ mit $v(x)=0$ ist auch
      $x^{-1}\in R$, da $v(x^{-1})=-v(x)=0\geq 0$.
    \item $\frakm_v$ ist nach denselben Argumenten wie in i)
      multiplikativ (bzgl. ganz $R$) und additiv (bzgl. sich selbst)
      abgeschlossen und liegt in $R$, ist demnach ein Ideal.
      Jedes echte Ideal darf keine Einheit enthalten, was nach ii) bedeutet,
      das alle Ideale nur Elemente in $\frakm_v$ enthalten
      können. $\frakm_v$ ist also maximal.
    \end{enumerate}
  \end{proof}
\end{Lem}

\begin{Kor}
  Ein Integritätsring $R$ ist genau dann ein Bewertungsring, wenn es
  eine auf $R$ definierte Bewertungsabbildung $v$ mit $v(x)\geq
  0\;\forall x\in R$ gibt.
  \begin{proof}
    \begin{itemize}
    \item[\enquote{$\Rightarrow$}] Eine entsprechende Abbildung
      existiert nach \ref{eigbewertungsring}.
    \item[\enquote{$\Leftarrow$}] Nach \ref{eigbewertungsring}ii)
      kann ich $v$ eindeutig auf den gesamten Quotientenkörper von
      $R$ fortsetzen, wobei dann offensichtlich $R=\{x\in\F\mid
      v(x)\geq 0\}$ gilt. Wende ich \ref{bewertungsringausbewertung}
      auf $\F$ mit $v$ an, erhalte ich, dass $R$ Bewertungsring ist.
    \end{itemize}
  \end{proof}
\end{Kor}

\begin{Bsp}
  Ein Beispiel eines Bewertungsringes ist $\Z_{(p)}$ mit der $p$-adischen
  Bewertung $v_p$ zu einem Primelement $p$, 
  wie sie in der Algebra eingeführt wurde: 
  $v_p$ liefert zu einem 
  $a=u\cdot p^{n}\cdot p_1^{n_1}\cdot p_2^{n_2}\dotsm\in\Z $ 
  den Exponenten $n$ von $p$ in der zu $a$ gehörigen Primfaktorzerlegung,
  was $\Z_{(p)}$ zu einem Bewertungsring macht.
\end{Bsp}

\subsection{Diskrete Bewertungen}
Mithilfe der $K$-Theorie können einige schöne Aussagen zu einer
speziellen Form bewerteter Körper getroffen werden, den diskret
bewerteten Körpern. Allerdings benötigen wir hierzu noch die wichtige
Eigenschaft, dass Elemente solcher Körper sich als Produkte aus einem speziellen
Primelement und Einheiten ihres Bewertungsrings darstellen lassen.

\begin{Def}
  Eine Bewertungsabbildung $v\colon\F\to\Gamma\cup\{\infty\}$
  auf einem Körper $\F$ heißt diskret, falls $\Gamma=\Z$ gewählt
  werden kann und $v$ surjektiv ist (insbes. $v^{-1}(1)\neq \varnothing$).
  Ein Bewertungsring heißt diskret, falls die zugehörige Bewertung
  diskret ist.
\end{Def}

\begin{Lem}\label{bewertungsringhir}
  Für einen Bewertungsring $R$ sind äquivalent
  \begin{enumerate}[i)]
  \item $R$ ist diskret
  \item $R$ ist Hauptidealring
  \end{enumerate}
  \begin{proof}
    Sei $R$ ein Bewertungsring zum Körper $\F$ mit Bewertung $v$.
    \begin{itemize}
    \item[i)$\Rightarrow$ii)] Sei $R$ diskret.
      Jetzt nutzen wir die Surjektivität (auch 
      Normiertheit genannt) von $v$ und wählen ein Element $\pi\in R$
      mit $v(\pi)=1$. Da $\Gamma=\Z$ ist und $\Z$ von 1 erzeugt wird,
      gilt für ein beliebiges Element $x\in R$, dass 
      $v(x)=k=k\cdot 1=1+1+\dotsb$. Damit ist 
      $v(\frac{x}{\pi^k})=v(x)-v(\pi^k)= k-k\cdot 1=0$, d.h. $x$ und 
      $\pi^k$ unterscheiden sich bloß um eine Einheit $u\in R$, s.d.
      $x=\pi^k \cdot u$.
      Nachdem alle Ideale nun allein Elemente der Form $\pi^k\cdot u$
      mit Einheit $u\in R$ enthalten, ist jedes Ideal $R\neq I\neq 0$ 
      erzeugt vom Element $\pi^k$ mit 
      $k=\min\{k\in\Z_{>0}\mid \exists x=\pi^k\cdot u\in I, u\in R^\times\}$
      der kleinste Exponent, der unter Elementen in $I$ auftritt.
      Es folgt, dass alle Ideale von einem Element erzeugt sind und
      $R$ ist Hauptidealring.
    \item[ii)$\Rightarrow$i)] Im Hauptidealring $R$ schreibe das 
      Maximalideal $\frakm=(\pi)$. Dann kann eine Bewertung
      konstruiert werden, die zu $x\in R$ den maximalen Exponenten $k$ liefert, 
      für den $a\in (\pi^k)$ gilt.
      Die Wohldefiniertheit folgt daraus, dass
      $\bigcap_{k=1}^{\infty}\pi^k=0$ ist in einem Hauptidealring.
    \end{itemize}
  \end{proof}
\end{Lem}


\begin{Bem}\label{darstellungdbk}
  \begin{enumerate}[a)]
  \item Ein diskreter Bewertungsring ist nach \ref{bewertungsringhir}
    lokaler Hauptidealring mit seinem Maximalideal als einzigem 
    Primideal (in Hauptidealringen sind alle Primideale maximal),
    welches von der Form $\frakm_v=(\pi)$ für ein Primelement $\pi$ ist.
    Die Primelemente in $R$ sind folglich genau die $p\in R$ mit $v(p)=1$, 
    genau die mit $\pi$ assoziierten Elemente.
    Wie im Beweis von \ref{bewertungsringhir} kann jedes Element $0\neq x\in R$
    in der Form $x=\pi^k\cdot u$, $k\in\N_0$, $u\in R^\times$ 
    geschrieben werden.
    Erweitert man $v$ auf den Quotientenkörper, erhält man
    \begin{gather*}
      \forall 0\neq x\in \F\colon\;
      \exists k\in\Z, u\in R^\times\colon\; x=\pi^k\cdot u
    \end{gather*}
  \item Die Bewertung $v\colon R\to\Z\cup\{\infty\}$ zu einem diskreten
    Bewertungsring ist aufgrund der Normierung eindeutig.
    Es gibt also zu einem Körper $\F$ einen Isomorphismus
    \begin{gather*}
      \{v\colon \F\to\Z\cup\{\infty\} 
      \mid \text{$v$ diskrete Bewertung}\}
      \cong 
      \{R\subset \F\mid \text{$R$ diskreter Bewertungsring auf $\F$}\}
    \end{gather*}
  \end{enumerate}
\end{Bem}


% \begin{Wdh}[Diskret bewertete Körper]
%   Ein diskret bewerteter Körper ist ein Körper $\F$ mit einer diskreten
%   Bewertungsabbildung $ v\colon \F\to \Z$.
%   Die Menge $R\coloneqq\left\{x\in \F\mid  v(x)\geq 1\right\}$ bildet
%   einen Bewertungsring mit den Eigenschaften
%   %   einen Integritätsring (tatsächlich sogar Hauptidealring) mit den Eigenschaften
%   \begin{enumerate}[i)]
%   %   \item $\forall x\in\F\colon x\in\R$ oder $x^{-1}\in\R$
%   \item $\Quot(R)=\F$
%   \item Die Einheitengruppe ist $R^\times\coloneqq\left\{x\in\F\mid v(x)=0\right\}$
%   \item $R$ ist lokal mit maximalem Ideal
%     $\frakm_v\coloneqq\left\{x\in\F\mid v(x)>1\right\}$
%   \item Die Primelemente von $R$ sind genau die Elemente $\pi\in\R$ mit $v(\pi)=1$.
%   \end{enumerate}
%   %   
%   %   lokalen Ring mit maximalem Ideal
%   %   $\frakm_v\coloneqq\left\{x\in\F\mid v(x)>1\right\}$,
%   %   Einheitengruppe $R^\times\coloneqq\left\{x\in\F\mid v(x)=1\right\}$
%   %   und den Eigenschaften $\Quot(R)=\F$ sowie 
%   %   $x\in R$ oder $x^{-1}\in R$ für alle $x\in\F$.
%   %   Ein Ring mit diesen Eigenschaften wird Bewertungsring genannt.
%   Der Körper $R/\frakm_v$ zum lokalen Ring $R$ heißt Restklassenkörper
%   von $R$.
% \end{Wdh}

\section{Das Zahme Symbol in der $K$-Theorie}
Im Folgenden sei $\F$ ein diskret bewerteter Körper mit
Bewertungsabbildung $v\colon \F\to\Z\cup\{\infty\}$ und
$\overline{\F}$ der Restklassenkörper $R/\frakm_v$ 
des zugehörigen Bewertungsrings $R=\left\{x\in\F\mid v(x)\geq0\right\}$
über dessen Maximalideal $\frakm_v=\left\{x\in\F\mid v(x)>0\right\}=(\pi)$.

\begin{Satz}\label{zahmessymbolhom}
  Zu einem diskret bewerteten Körper $(\F,v)$ gibt es für alle $n\geq 1$
  einen eindeutigen Gruppenhomomorphismus 
  $\partial\colon \KnF\to \K_{n-1}(\bF)$,
  der für alle Primelemente $\pi\in R$ und 
  Einheiten $u_1,\dotsc, u_n\in R$ erfüllt
  \begin{gather*}
    \partial(\{\pi,u_1,\dotsc,u_n\})=\{\ov u_1, \dotsc, \ov u_n\}
  \end{gather*}
\end{Satz}


\begin{Bem}\label{bemzahmessymbolhom}
  \begin{enumerate}[a)]
  \item   $\partial$ muss surjektiv sein, da $\ov\F^\times\subset \Fx$
    Untergruppe ist.
  \item   Mit der Relation $u_1=\frac{\pi u_1}{\pi}$ 
    für eine Einheit $u_1\in R^\times$
    wird für bel. $u_1,\dotsc, u_n\in R^\times$
    \begin{align*}
      \partial(\{u_1,\dotsc,u_n\})
      &= \partial(\{\pi u_1\cdot \pi^{-1},u_2,\dotsc,u_n\})\\
      &= \partial(\{\pi u_1,u_2,\dotsc,u_n\})\m 
        \partial(\{\pi,u_2,\dotsc,u_n\})\\
      &\overset{\mathllap{\text{Def. }\partial}}{
        \underset{\mathllap{\pi,\pi u\text{ prim}}}{=}}
        \{\ov u_2,\dotsc, \ov u_n\} \m \{\ov u_2,\dotsc, \ov u_n\}
        = 0
    \end{align*}
  \item Für $n=1$ erfüllt $\partial=v\colon \F\to \Z\cup\{\infty\}$ die Bedingung.
    Denn hier ist sie äquivalent dazu, dass ein Primelement 
    $\pi\in R\subset \Fx$ auf die (einzige) Einheit 1 
    geschickt wird und eine Einheit $u\in R\subset \Fx$ auf 0,
    was eine diskrete Bewertung tut, wie wir gesehen haben.
  \end{enumerate}
\end{Bem}

\begin{proof}[Eindeutigkeit zu \ref{zahmessymbolhom}]
  Um die Eindeutigkeit zu zeigen, benötigen wir die Erkenntnis aus der
  Bewertungstheorie, dass für ein fix gewähltes Primelement $\pi\in
  R\subset \F$ jedes Element $x\in\Fx$ 
  dargestellt werden kann als $x=\pi^k\cdot u$ mit einem $k\in\Z,
  u\in R^\times$, siehe \ref{darstellungdbk}.
  Dies kann auf $\KnF$ übertragen werden in dem Sinne, dass 
  $\KnF$ von Elementen der Form 
  $\overline{l(\pi)^kl(u_{k+1})\dotsm l(u_n)}
  =\{\pi,\dotsc, \pi, u_{k+1},\dotsc,u_n\}
  =\{\pi\}^k\otimes\{u_{k+1},\dotsc,u_n\}\in\KnF$ mit $\pi$ prim und
  $u_i$ Einheit in $R$ erzeugt ist, wie wir jetzt zeigen.
  Wähle also ein fixes Primelement $\pi\in\R\subset\Fx$.
  und sei $\xi\coloneqq\{a_1,\dotsc, a_n\}\in\KnF\subset\KF$ mit
  $a_1,\dotsc,a_n\in\Fx$. Dann gibt es Darstellungen
  $a_n=\pi^{k_n}\cdot u_n$ mit $k_n\in\Z,u_n\in R^\times$ wie
  oben. $\xi$ lässt sich nun schrittweise additiv schreiben
  \begin{align*}
    \xi&=\{a_1,\dotsc,a_n\}=\{\pi^{k_1}u_1, \dotsc, \pi^{k_n}u_n\}\\
       &=\{\pi u_1, \dotsc, \pi^{k_n}u_n\}
         \p\{\pi^{k_1-1}u_1, \dotsc, \pi^{k_n}u_n\}\\
       &=\dotsb
         =k_1\cdot \{\pi u_1,\pi^{k_2}u_2, \dotsc, \pi^{k_n}u_n\}\\
       &=\dotsb 
         =k_1\dotsm k_n\{\pi u_1, \dotsc, \pi u_n\}
  \end{align*}
  Das Element $\{\pi u_1, \dotsc, \pi u_n\}$ lässt sich auch wieder
  auseinanderziehen als
  \begin{align*}
    \{\pi u_1, \dotsc, \pi u_n\} 
    &= \{\pi,\pi u_2,\dotsc,\pi u_n\}\p\{u_1,\pi u_2,\dotsc,\pi u_n\}\\
    &= \{\pi,\pi,\pi u_3,\dotsc,\pi u_n\}
      \p\{u_1,\pi,\pi u_3,\dotsc,\pi u_n\}\\
    &\p\{\pi,u_2,\pi u_3,\dotsc,\pi u_n\}
      \p\{u_1,u_2,\pi u_3,\dotsc,\pi u_n\}\\
    &=\dotsb=\sum_{\mathclap{(i_1,\dotsc,i_n)\in\{0,1\}^n}}
      \{\pi^{i_1}u_1^{1-i_1},\dotsc,\pi^{i_n}u_n^{1-i_n}\}\\
  \end{align*}
  Wegen \ref{identitaetmal} können in einem der Summenglieder nur mit
  Vorzeichenwechsel die $\pi$ im Produkt nach vorne geschoben werden,
  sodass $\xi$ zu einer Summe aus oben beschriebenen Elementen wird.
  Nachdem $\partial$ ein Homomorphismus ist, genügt es, $\partial$ auf
  Elementen der Form $\{\pi\}^k\otimes\{u_{k+1},\dotsc,u_{n}\}$ zu
  betrachten.

  Für $n=1$ ist nach einem Homomorphismus $\partial\colon \Fx\to\Z$
  gefragt, der genau die Einheiten auf $0$ schickt und Primelemente nicht.
  Diese Eigenschaft liefert mit der Erweiterung $\partial(0)=\infty$
  genau eine $\Z$-wertige Bewertung auf $\F$ mit Bewertungsring
  $R$. Diese muss nach \ref{bemzahmessymbolhom} surjektiv sein
  und wird so zur diskreten Bewertung.
  Damit ist sie bereits eindeutig durch ihren Bewertungsring bestimmt 
  nach \ref{darstellungdbk}.

  Für $n>1$ wandelt man zuerst den Ausdruck $\{\pi\}^k$ durch
  sukzessives Anwenden von \ref{identitaetquadrat} um in
  \begin{gather*}
    \{\pi\}^k
    \overset{\ref{identitaetquadrat}}{=} \{\pi\}^{k-1}\{-1\}
    \overset{\ref{identitaetquadrat}}{=} \{\pi\}^{k-2}\{-1\}^2
    =\dotsb=\{\pi\}\otimes\{-1\}^{k-1}
  \end{gather*}
  Hiermit gilt für jeden Erzeuger
  $\{\pi\}^k\otimes\{u_1,\dotsc,u_n\}\in\KnF$ von $\KnF$
  \begin{align*}
    \partial\left(\{\pi\}^k\otimes\{u_1,\dotsc,u_n\}\right)
    &= \partial\left(
      \{\pi\}\otimes\{-1\}^{k-1}\otimes\{u_1,\dotsc,u_n\}
      \right)\\
    &\overset{\mathllap{\text{Def. $\partial$}}}{=}
    \{\ov {-1}\}^{k-1}\otimes\{\ov u_1,\dotsc,\ov u_n\}
  \end{align*}
  Also ist $\partial$ bereits auf jedem Erzeuger von $\KnF$ eindeutig
  bestimmt, was die Eindeutigkeit auf ganz $\KnF$ liefert.
\end{proof}

\begin{proof}[Existenz]
  Wir werden $\partial$ explizit konstruieren.
  Von der Idee her möchten wir gerne – wieder zu unserem fix gewählten
  Primelement $\pi\in R$ – ein Element $\{\pi^ku\}\in\KoF, u\in R^\times,$
  in seinen \enquote{$\pi$-Anteil} und seinen 
  \enquote{Einheitsanteil} zerlegen. 
  Dann können wir die Gruppenisomorphie
  $\K_1(\bF)\cong\bF^\times=(R/\frakm_v)^\times\cong R^\times$
  ausnutzen und den \enquote{Einheitsanteil} als Element aus $\K_1(\bF)$
  schreiben. Durch geschickte Umformung liefert uns dies dann den
  gesuchten Ringhomomorphismus, wie wir gleich sehen werden.
  % Anschaulich zerlegt $\psi$ ein Element $\{\pi^ku\in\}\KoF\cong\Fx$ in 
  % seinen $\pi$- und seinen Einheitsanteil. Da die Gruppenisomorphie
  % $\K_1(\bF)\cong\bF^\times\cong (R/\frakm_v)^\times\cong R^\times$
  % herrscht, kann ich meinen 
  
  Um unseren \enquote{$\pi$-Anteil} sauber darstellen zu können,
  adjungieren wir an $\KbF$ ein Element $\xi$, das wir über
  folgende Bedingungen definieren:
  \begin{enumerate}[(1)]
  \item $\xi\cdot \alpha=(-1)\alpha\xi$ für alle $\alpha\in\KbF$
  \item $\xi^2=\xi\{\ov{-1}\}$
  \end{enumerate}
  Damit erhalten wir einen erweiterten Ring $\KbF[\xi]$, der als
  $\KbF$-Modul die Basis $\{ \xi, 1\in\Z\}$ hat (also frei ist über
  $\KbF$). $\xi$ entspricht anschaulich einer Erweiterung um $\pi$.

  Als nächstes zeigen wir, dass die Zuweisung ($u\in R^\times$)
  \begin{gather*}
    \psi\colon\{\pi^ku\}\mapsto i\xi \p \{\ov u\}
    =\begin{pmatrix} i\\\{\ov u\} \end{pmatrix}
    \qquad \psi(1\in\Z)=1\in\Z=\begin{pmatrix}0\\1\in\Z\end{pmatrix}
  \end{gather*}
  sich eindeutig zu einem Ringhomomorphismus
  $\psi\colon\KF\to\KbF[\xi]$ fortsetzen lässt.
  $\psi$ ist die gewünschte Zerlegung mit $\psi(\{\pi\})=\xi$ 
  und $\psi(\{u\})=\{\ov u\}\in\K_1\bF$ für
  eine Einheit $u\in R^\times$.
  
  Setze als ersten Schritt die Fortsetzung von $\psi$ auf ganz $\KF$
  ($\{a_1,\dotsc,a_n\}\in\KF$)
  \begin{gather*}
    \psi(\{a_1,\dotsc,a_n\})
    \coloneqq \left\{\psi(\{a_1\})\right\}\otimes\dotsb
      \otimes \left\{\psi(\{a_n\})\right\}
  \end{gather*}
  Damit ist die Homomorphismuseigenschaft
  $\psi(\{x\}\otimes\{y\})=\psi(\{x\})\otimes\psi(\{y\})$
  bereits gegeben.

  Die Wohldefiniertheit ist klar, da sich wie in \ref{darstellungdbk}
  jedes Element aus $\Fx$ als $\pi^k u$ darstellen lässt und
  $\KF$ das Erzeugnis aus Produkten und Summen von $\{x\}, x\in\Fx$
  ist. Somit ist $\psi$ für jedes Element eindeutig bestimmt.

  Betrachte Elemente $x,y\in\Fx, x=\pi^iu, y=\pi^jv$ mit Einheiten
  $u,v\in\R^\times$. 
  Dann gilt $x\cdot y=\pi^iu\cdot\pi^jv=\pi^{i+j}(uv)$  und
  \begin{align*}
    \psi(\{x\}\p\{y\})
    &=\psi(\{x\cdot y\})=\psi(\{\pi^{i+j}(uv)\})
      =\begin{pmatrix}(i+j)\\\{\ov{uv}\}\end{pmatrix}\\
    &=\begin{pmatrix}(i+j)\\\{\ov u \}\p \{\ov v\}\end{pmatrix}
    = \begin{pmatrix}i\\\{\ov u\}\end{pmatrix}
    \p\begin{pmatrix}j\\\{\ov v\}\end{pmatrix}
    =\psi(\{\pi^iu\})\p \psi(\{\pi^jv\})\\
    &= \psi(\{x\})\p\psi(\{y\})
  \end{align*}
  Beachte hierbei, dass für eine Einheit $u\in R^\times$ das Urbild
  der Projektion $\ov u\in\bF=R/\frakm_v$ eindeutig gleich $u$ ist,
  da die multiplikativen Gruppen $\bF$ und $R^\times$ isomorph sind.

  Jetzt haben wir einen Ringhomomorphismus von $\KF$ nach $\KbF[\xi]$.
  Diesen wollen wir nun durch Umformung ein wenig umformulieren
  ($a_i=\pi^{k_i}u_i\in\Fx$):
  \begin{gather*}
    \psi(\{a_1,\dotsc,a_n\})
    =\bigotimes_{1\leq i\leq n}(k_i\xi\p\{\ov u_i\})
    \overset{\text{Ausmult.}}{=}
    \{\ov u_1,\dotsc,\ov u_n\}\p \chi
  \end{gather*}
  wobei $\chi\in\KnbF[\xi]$ eine Summe aus Produkten von $\xi$ und den
  $\{\ov u_i\}$ ist.
  Wie beim  Beweis der Eindeutigkeit lassen sich die Glieder 
  von $\chi$ mithilfe von \ref{identitaetmal}
  (und der Forderung an $\xi$) umformen, so dass sie alle die Form
  $\pm\xi^k\{v_{k+1},\dotsc,v_n\}$ haben mit $v_i=\ov u_j$ für ein 
  $j\in\{1,\dotsc,n\}$ und $0\leq k\leq n$.
  Mit der Forderung $\xi^2=\xi\otimes\{\ov {-1}\}$ kann ich wieder 
  wie im Eindeutigkeitsbeweis
  alle Ausdrücke mit $k\neq 0$ in die Form
  $\xi\otimes\{\ov{-1}\}^{k-1}\{v_{k+1},\dotsc,v_n\}$ bringen.
  Also haben alle Summenglieder von $\chi$ die Form
  $\xi\otimes\{v_1,\dotsc, v_n\}$ mit $v_i\in\bF$ und ich kann
  schreiben
  \begin{gather*}
    \psi(\{\pi^{k_1}u_1,\dotsc,\pi^{k_n}u_n\}) 
    = \{\ov u_1,\dotsc,\ov u_n\}\p \xi\chi'
  \end{gather*}
  wobei $\chi'\in\KnbF$ die Summe der Faktoren $\{v_1,\dotsc,v_n\}$
  von oben ist.
  Ich kann also schreiben mit $\gamma(\{a_1,\dotsc,a_n\})=\{\ov
  u_1,\dotsc,\ov u_n\}$
  und $\partial(\{a_1,\dotsc,a_n\})=\chi'$ wie oben
  \begin{gather*}
    \psi(\alpha) = \gamma(\alpha)\p \xi\partial(\alpha)
    \quad \forall\alpha\in\KnF
  \end{gather*}
  woraus schon folgt, dass $\gamma,\delta\colon\KnF\to\KnbF$ 
  Ringhomomorphismen sind.
  Dieses $\partial$ erfüllt tatsächlich die Anforderungen von oben,
  denn für $\alpha=\{\pi,u_1,\dotsc, u_n\}$ ist
  \begin{align*}
    \psi(\alpha)
    &=\psi(\pi)\otimes\psi(u_1)\otimes\dotsb\otimes\psi(u_n)\\
    &=\xi\otimes\{\ov u_1\}\otimes\dotsb\{\ov u_n\}\\
    &=\xi\otimes\{\ov u_1,\dotsc,\ov u_n\}\\
    &=(0\otimes\{\ov u_1,\dotsc,\ov u_n\})
      \p\xi\{\ov u_1,\dotsc,\ov u_n\}\\
    &=\gamma(\alpha) \p \xi\partial(\alpha)
  \end{align*}
  Daraus folgt bereits
  $\partial(\{\pi,u_1,\dotsc,u_n\})=\{\ov u_1,\dotsc,\ov u_n\}$.  
\end{proof}

\begin{Kor}
  Zu einem Bewertungskörper $(\F,v)$ gibt es zu einem fixen
  Primelement $\pi\in R$ einen eindeutigen Ringhomomorphismus 
  $\gamma\colon\KF\to\KbF$ der für alle Einheiten $u\in R^\times$, 
  $k\in\Z$ erfüllt
  \begin{gather*}
    \gamma(\{\pi^ku\})=\{\ov u\}
  \end{gather*}

  \begin{proof}
    Die Existenz folgt aus dem Existenzbeweis von
    \ref{zahmessymbolhom}, da das dort definierte $\gamma$ die
    Forderung erfüllt.
    Demnach ist Wohldefiniertheit und
    Ringhomomorphismuseigenschaft gegeben.
    Die Eindeutigkeit liefert wie in \ref{zahmessymbolhom}
    die eindeutige Darstellung von Elementen in $\KoF=\Fx$ 
    als Produkt aus $\pi$ und Einheit.
  \end{proof}
\end{Kor}


% ------------

\section{Die spaltende exakte Milnor-Folge}
\cite[][Theorem 2.3]{milnor}


% ------------------------------------------------------------------

\printindex
% \listoffigures		
% \listoftables

\nocite{*}
\printbibliography


\end{document}

% % % % % % % % % % % % % % % % % % % % % %
%
%     AUSARBEITUNG 
%  zum Seminar K-Theory
%       im SS2015
% Universität Regensburg
%  
%     THEMA: Das zahme Symbol und diskrete Bewertungsringe
%            (K2 eines endlichen Körpers veschwindet, diskrete
%             Bewertungsringe und das zahme Symbol [4, Lemma 2.1], die
%             spaltende exakte Milnor-Folge [4, Theorem 2.3].)
%
%
%	Config Datei: Definitionen etc.
%
% % % % % % % % % % % % % %


% ESSENTIAL PACKAGES
% ---------------------------------------------------------------
\usepackage[T1]{fontenc}
\usepackage[utf8]{inputenc}
\usepackage{babel}

\usepackage{lmodern}
\usepackage{microtype}

% EXTRA PACKAGES
% ---------------------------------------------------------------
%% pagestyle
\usepackage{scrlayer-scrpage}
\usepackage{enumerate}
\usepackage{csquotes}

%% maths
\usepackage{amsmath}
\usepackage{mathtools}
\usepackage{amsthm}
\usepackage{amssymb}
\usepackage{dsfont}

%% index/bibliography
\usepackage{makeidx}
\makeindex
\usepackage[date=iso8601,backend=biber,style=alphabetic]{biblatex}
\bibliography{k-theory.bib}

% commutative diagrams
\usepackage{tikz-cd}

% equation counter reset at sections
\usepackage{chngcntr}
% \counterwithin*{equation}{section}
% \renewcommand{\theequation}{\thesection.\arabic{equation}}
\numberwithin{equation}{section}

% (nicer)tables
\usepackage{booktabs}
% hyperlinks
\usepackage{hyperref}
\hypersetup{%
pdftitle={Das Zahme Symbol und diskrete Bewertungsringe},%
pdfauthor={Gesina Schwalbe}%
}


% DEFINITIONS
% ---------------------------------------------------------------
% math (theorems)
\theoremstyle{definition}
\newtheorem{Def}{Definition}[section]
\newtheorem{Bsp}[Def]{Beispiel}
\newtheorem{Not}[Def]{Notation}
% \theoremstyle{remark}
\newtheorem{Bem}[Def]{Bemerkung}
\newtheorem{Wdh}[Def]{Wiederholung}
\theoremstyle{plain}
\newtheorem{Satz}[Def]{Satz}
%\newtheorem*{satz}{Satz}
\newtheorem{Lem}[Def]{Lemma}
\newtheorem{Kor}[Def]{Korollar}
\newtheorem{Fol}[Def]{Folgerung}


% math operators/symbols
\DeclareMathOperator{\Const}{const}
\DeclareMathOperator{\Ker}{ker}
\DeclareMathOperator{\im}{im}
\DeclareMathOperator{\Char}{char}
\DeclareMathOperator{\Ord}{ord}
\DeclareMathOperator{\Quot}{Q}
\DeclareMathOperator{\id}{id}
\DeclareMathOperator{\pr}{pr}
%\DeclareMathOperator{\K}{K}
\newcommand{\K}{K}

\newcommand{\N}{\mathds{N}}
\newcommand{\Z}{\mathds{Z}}
\newcommand{\Q}{\mathds{Q}}
\newcommand{\R}{\mathds{R}}
\newcommand{\C}{\mathds{C}}
\newcommand{\F}{\mathds{F}}
\newcommand{\bF}{\overline{\F}}

\newcommand{\KF}{\K_\ast(\F)}
\newcommand{\KbF}{\K_\ast(\bF)}
\newcommand{\KnF}{\K_n(\F)}
\newcommand{\KnbF}{\K_n(\bF)}
\newcommand{\KoF}{\K_1(\F)}
\newcommand{\Fx}{\F^\times}
\newcommand{\Rx}{R^\times}
\newcommand{\bFx}{\bF^\times}
\newcommand{\Ft}{\F(t)}
\newcommand{\Ftpi}{\F[t]/(\pi)}
\newcommand{\KnFt}{\K_n(\Ft)}
\newcommand{\KnFtpi}{\K_n(\Ftpi)}

\newcommand{\ov}{\overline}

\usepackage{MnSymbol}
\newcommand{\p}{\diamondplus}%{\bigplus}%{\boldsymbol +}}
\newcommand{\m}{\boldsymbol -}

\newcommand{\calA}{\mathcal{A}}
\newcommand{\calK}{\mathcal{K}}
\newcommand{\frakm}{\mathfrak{m}}

\newcommand*{\disoarrow}{\arrow[d,"\rotatebox{90}{$\thicksim$}"]}

\newcommand{\somenote}[1]{\marginpar{#1}}

% STYLE SETTINGS
% ---------------------------------------------------------------

\ihead{Gesina Schwalbe}
\pagestyle{scrheadings}

\title{Das zahme Symbol und diskrete Bewertungsringe}
\subject{Ausarbeitung}
\subtitle{im Seminar K-Theorie von Prof. Dr. Moritz Kerz}
%\place{Universität Regensburg}
\author{Gesina Schwalbe}
\date{01.07.2015}


%%% Local Variables:
%%% mode: latex
%%% TeX-master: "k-theory_script"
%%% End:
